\documentclass[12pt, letterpaper]{article}
\usepackage[titletoc,title]{appendix}
\usepackage{color}
\usepackage{booktabs}
\usepackage{caption}
\newcommand\fnote[1]{\captionsetup{font=small}\caption*{#1}}
\usepackage{float}

\usepackage[usenames,dvipsnames,svgnames,table]{xcolor}
\definecolor{dark-red}{rgb}{0.75,0.10,0.10} 
\usepackage[margin=1in]{geometry}
\usepackage[linkcolor=blue,
            colorlinks=true,
            urlcolor=blue,
            pdfstartview={XYZ null null 1.00},
            pdfpagemode=UseNone,
            citecolor={blue},
            pdftitle={Optimal Sequence}]{hyperref}

\usepackage{multibib}
\usepackage{geometry} % see geometry.pdf on how to lay out the page. There's lots.
\geometry{letterpaper}               % This is 8.5x11 paper. Options are a4paper or a5paper or other... 
\usepackage{graphicx}                % Handles inclusion of major graphics formats and allows use of 
\usepackage{amsfonts,amssymb,amsbsy}
\usepackage{amsxtra}
\usepackage{natbib}

\usepackage{longtable}
\usepackage{array}
\usepackage{multirow}
\usepackage{wrapfig}
\usepackage{colortbl}
\usepackage{pdflscape}
\usepackage{tabu}
\usepackage{threeparttable}
\usepackage{threeparttablex}
\usepackage[normalem]{ulem}
\usepackage{makecell}
\usepackage{verbatim}
\setcitestyle{round,semicolon,aysep={},yysep={;}}
\usepackage{setspace}             % Permits line spacing control. Options are \doublespacing, \onehalfspace
\usepackage{sectsty}             % Permits control of section header styles
\usepackage{lscape}
\usepackage{fancyhdr}             % Permits header customization. See header section below.
\usepackage{url}                 % Correctly formats URLs with the \url{} tag
\usepackage{fullpage}             %1-inch margins
\usepackage{multirow}
\usepackage{rotating}
\setlength{\parindent}{3em}
\usepackage{subcaption}
\usepackage[T1]{fontenc}
\usepackage{bm}
\usepackage{libertine}
\usepackage{amsmath}
\DeclareMathOperator*{\minimize}{minimize}

\usepackage{chngcntr}

\title{\Large{Optimal Sequence in Which to Service Orders}}
\author{Ken Cor\thanks{Ken can be reached at \href{mailto:mcor@ualberta.ca}{\footnotesize{\texttt{mcor@ualberta.ca}}}} \and Gaurav Sood\thanks{Gaurav can be reached at: \href{mailto:gsood07@gmail.com}{\footnotesize{\texttt{gsood07@gmail.com}}}}\vspace{.5cm}}



\date{\vspace{.5cm}\normalsize{\today}}

\begin{document}
\maketitle

\begin{comment}

setwd(paste0(githubdir, "optimal_sequence/ms/"))
tools::texi2dvi("optimal_sequence.tex", pdf = TRUE, clean = TRUE) 
setwd(githubdir)

\end{comment}

Given the cost of a set of tickets on any given day is defined by matrix $c_{ij}$ where $i$ denotes ticket and $j$ denotes day. On which day should each ticket be purchased in order to minimize the total cost to purchase the tickets. 

The maximum number of tickets that an employee can purchase in a day is $m$. And the daily wage of a person is $w$. And the number of people staffed is $n_j$.

Decision Variables:

$$
$x_{ij} = 
\begin{cases}
      1 & \text{if ticket t is purchased on day j} \\
      0 & \text{otherwise}
\end{cases}
$$

Minimize the objective Function:

\begin{align*}
\Sigma_j \Sigma_i (c_{ij} + h*e) x_{ij}
\end{align*}

subject to the following constraints:

\begin{align*}
    \Sigma_j x_{ij} == 1\\
    p_{ij} == 1\\
    \Sigma_i h * x_{ij} \leq v_j
\end{align*}

Given $c_{ij}$, $p_{ij}$, $w$, $$, ...

$$
$p_{ij} = 
\begin{cases}
      1 & \text{if ticket t can be purchased on day j} \\
      0 & \text{otherwise}
\end{cases}
$$\\


What is the optimal order in which to service orders assuming a fixed budget?

Let's assume that we have to service orders $o_1,...,...o_n$, with the $n$ orders iterated by $i$. Let's also assume that for each service order, we know how the costs change over time. If we receive service order $o_i$ at time $t$, we expect the cost to be $c_{it}$ at $t$, $c_{it+1}$ at $t+1$, etc. Each service order also has an expiration time, $j$, after which the order cannot be serviced. The cost at expiration time, $j$, is the cost of failure and denoted by $c_{ij}$. Expectedly, $j \geq t$.

For simplicity, let's also assume that time is discrete and portioned in days. It means that the cost of servicing an item on the same day---$c_{it}$---don't change depending on when the item was serviced during the day.

New service items arrive each day. And we re-prioritize the queue every day.

Let's denote the time it takes to finish a work item by $w_i$ and our budget (total person hours) for each day, $W_t$. For the moment, let's assume $w_i$ to be the same for all $i$---it takes the same amount of time to clear an item.

We work till $\Sigma_{it} w_{it} = W_t$ or till the queue for the day runs out. 

The first question is if we need to punt any items to tomorrow. If $\Sigma_{it} w_{it} \leq W_t$, there is nothing more to do. If not, we need to make some choices. The optimal sequence of servicing orders is determined by expected losses---first service the order where the expected loss is the greatest. This leaves us with the question of how to estimate expected loss at time $t$. We need to answer two questions: 1. if we don't service the order today, what are our chances of getting to it the next day, the day after, etc., till $j$, or $p_{it}$, and 2. what is the cost we will have to bear if we postpone servicing the order to $t+1$ to $j$, or $c_{it}$. To get the total cost of deferring, we just need to first estimate what we expect to pay if we deferred: multiply $p_{it}$ with $c_{ij}$ over all $t$ from $t+1$ to $j$. And then subtract what we expect to pay if we deferred from what we would pay at time $t$. So framed, the expected loss for order $i$ at time $t$ = $c_{it} - \Sigma_{t+1}^{j} p_{it} * c_{it}$.

However, determining $p_{it}$ is not straightforward. New items are added to the queue every day. On the flip side, we also get to re-prioritize every day. The question then is if we will get to the item $o_i$ at $t+1$? (It means $p_{it}$ is 0 or 1.) For that, we need to forecast the length of the queue tomorrow. We can do this with a model. Let's assume, for now, that the model is perfect and we can forecast with perfect accuracy. 

If tomorrow's queue length including items from today is less than $W_{t+1}$, then the cost of deferral is simply $c_{it+1}$. But if the queue will take more than $W_{t+1}$, then we need to calculate expected losses for all the items tomorrow to figure out which make the cut. To do that, we need to again start afresh. Hence, this becomes an infinite process.

To make it tractable, we need to make some assumptions. In some cases, one reasonable assumption is to assume that $c_{ij}$ is very high. This effectively would mean that the items are always cleared by $j-1$. This kind of queue clearing in cases where we don't have enough days where we can pick the slack can eventually become a system where the only items that are cleared are ones that are about to expire. 

Another reasonable assumption in some cases would be to assume that the cost of deferring for a day give us the right rank order of costs. Then the optimal thing to do is to clear items based on one day out costs.

A more rational approach to queuing would be to budget in a profitable way. Forecast the size of the queue and the items in the queue. Forecast the cost of waiting one day for all items. Sequence the costs in a way that we clear the most costly. And then buy as many person-hours as make sense to allow for clearing the remaining items the same day.

\end{document}
